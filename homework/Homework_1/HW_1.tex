%выбор типа документа -- статья; в квадратных скобках
%выбирается размер шрифта
\documentclass[12pt]{article}
%выбор размера полей
\usepackage[margin=1in]{geometry}
%здесь указаны пакеты без опций в квадратных скобках
\usepackage{setspace,float,amsmath,amsxtra,hyperref,authblk,amssymb,graphicx,multirow,xcolor,lscape,rotating,url,fancyhdr,multirow,tabularx,booktabs, amsthm}
\usepackage[utf8]{inputenc}
%подгрузка русского алфавита
\usepackage[russian]{babel}
\pagestyle{fancy}
\lhead{Методология и Методы Исследований в Социальных Науках}
\rhead{Евгений Седашов, 2024}
\hypersetup {
colorlinks=true,
linkcolor=cyan,
urlcolor=blue,
filecolor=magenta
}
\linespread{1.3}
%эти команды даны опционально для демонстрации
\newtheorem{thm}{Theorem}[section]
\newtheorem{lem}[thm]{Lemma}
\newtheorem{exr}[thm]{Exercise}
\makeatletter
\renewcommand*\env@matrix[1][*\c@MaxMatrixCols c]{%
	\hskip -\arraycolsep
	\let\@ifnextchar\new@ifnextchar
	\array{#1}}
\makeatother
\begin{document}
\begin{center}
\huge \textbf{Домашнее Задание 1} \\
\normalsize Дедлайн -- 20 ноября.  
\end{center}

\textbf{Задание 1.}

Дайте операционные определения следующим конструктам: 

\begin{itemize}
\item Протестный потенциал.
\item Способность к чтению. 
\item Политическая идеология.
\item Тревожность. 
\item Экономическое неравенство. 
\item Политическое влияние. 
\end{itemize}

Для любых двух из пяти перечисленных выше конструктов придумайте гипотезу, где конструкт выступает в качестве зависимой или независимой переменной.  Дайте короткое (максимум 1 абзац) теоретическое обоснование гипотезы. 

\textbf{Задание 2.}

a) Предположим,  что Вы тестируете следующую гипотезу: ``Люди с высшим образованием (бакалавр и выше) более склонны испытывать состояние тревожности''.   Обоснуйте данную гипотезу логически.  Как должна выглядеть описательная таблица (или график) в случае,  если гипотеза подтверждается? 

b) Предположим,  что Вы тестируете следующую гипотезу: ``Увеличение экономического неравенства ведёт к уменьшению межличностного доверия в обществе''.  Обоснуйту данную гипотезу логически.  Как должна выглядеть описательная таблица (или график) в случае,  если гипотеза подтверждается? 

с) Предположим,  Вы пытаетесь проверить,  связано ли восприятие людьми справедливости накопленного богатства с его источником.  К примеру,  люди с большей вероятностью будут воспринимать накопленное некоторым человеком богатством как справедливое,  если оно связано с некоторой улучшающей жизнь услугой (например, приложение для смартфона).  Попробуйте сформулировать экспериментальный дизайн для теста данной гипотезы. 

\textbf{Задание 3.}

Укажите,  к какому типу шкалы (номинальная,  ординальная,  интервальная) относятся указанные ниже переменные.  Кратко (максимум два предложения) объясните,  почему. 

a) Уровень ночной освещённости поверхности Земли (night time luminosity)

b) Длительность гражданской войны в днях

c) Наличие/отсутствие военного альянса между двумя государствами

d) Количество новорождённых в некотором регионе России

e) Уровень одобрения политики,  измеренный по шкале от 0 до 4,  где 0 -- полностью не одобряю,  4 -- полностью одобряю

f) Уровень инфляции

g) Явка на выборы (0 -- не ходил,  1 -- ходил).

\textbf{Задание 4.}

Опишите разницу между Between-Group и Within-Group Variance.  Какая связь этих понятий с понятиями Error Variance и Systematic Variance? 

\textbf{Задание 5 (бонусное).}

Прочитайте статью ``Prospect Theory: An Analysis of Decision under Risk'' (в папке).  Ответьте на следующие вопросы:

1) в чём заключается ключевой аргумент статьи? 

2) какие эмпирические доказательства авторы приводят в пользу предложенных аргументов? 

3) каков основной вывод статьи? 

\end{document}