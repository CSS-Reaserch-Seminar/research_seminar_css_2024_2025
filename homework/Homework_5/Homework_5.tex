%choose the document type and font
\documentclass[12pt]{article}
%margins
\usepackage[margin=1in]{geometry}
%packages
\usepackage{setspace,float,amsmath,amsxtra,hyperref,authblk,amssymb,graphicx,multirow,xcolor,lscape,rotating,url,fancyhdr,multirow,tabularx,booktabs, amsthm}
\usepackage[utf8]{inputenc}
\usepackage[russian]{babel}
\pagestyle{fancy}
\lhead{Методология и Методы Исследований в Социальных Науках}
\rhead{Евгений Седашов, 2024}
\hypersetup {
colorlinks=true,
linkcolor=cyan,
urlcolor=blue,
filecolor=magenta
}
\linespread{1.3}
%эти команды даны опционально для демонстрации
\newtheorem{thm}{Theorem}[section]
\newtheorem{lem}[thm]{Lemma}
\newtheorem{exr}[thm]{Exercise}
\begin{document}
\begin{center}
\huge \textbf{Домашнее Задание 5} \\
\normalsize Дедлайн -- 5 февраля \\
\end{center}
\textbf{Задание 1 (Python)}.

Создайте две переменные $X_1 \sim \mathcal{N}(3,1)$ (1 -- дисперсия) и $X_2 \sim \text{Poisson}(4)$ с 1000 наблюдений для каждой переменной.  Предположим,  что $Y$, $X_1$ и $X_2$ связаны следующим уравнением для генеральной совокупности: $$Y = 5 + 3 X_1 - 2 X_2 + u$$
Предположения Гаусса-Маркова верны.  Ответьте на следующие вопросы:

I.  Сгенерируйте 1000 случайных выборок переменной  $Y$ в соответствии с уравнением выше.  Ошибки $u$ следуют нормальному распределению со средним 0 и дисперсией 4.  Можно использовать генератор случайных чисел из библиотеки numpy. 

II.  Для каждой выборки оцените МНК-регрессию,  используя $X_1$ и $X_2$ в качестве независимых переменных.  Используйте OLS из библиотеки statsmodels,  примеры кода есть в ноутбуке из семинара Quasi-experiments. 

III.  Для каждой выборки сохраните получившиеся параметры модели (интерсепт плюс два коэффициента наклона).  Для каждого параметра модели посчитайте 95-процентные доверительные интервалы и проверьте,  включают ли (покрывают ли) данные интервалы параметры генеральной совокупности. 

IV.  Посчитайте средние арифметические по каждому из параметров.  Близки ли получившиеся значения параметрам модели в генеральной совокупности? Какое свойство МНК-оценок иллюстрируют данные результаты? 

V.  Посчитайте по каждому параметру модели процент доверительных интервалов,  покрывших реальные параметры модели в генеральной совокупности.  Вы должны получить числа,  близкие к 95 \%.  

VI.  Посчитайте дисперсию по каждому из параметров модели (дисперсию 1000 получившихся значений по каждому из параметров модели).  Как будет меняться данная дисперсия с ростом числа наблюдений? К какому значению она будет ассимтотически (количество наблюдений $\to \infty$) приближаться? О каком свойстве МНК-оценок здесь идёт речь? 

\textbf{Задание 2 (Python)}. 

Выполните задание C11 из учебника Вулдриджа,  Глава 4.  В дополнение к заданию,  представьте результаты регрессий из пунктов (i), (iii) и (iv)  в профессиональной регрессионной таблице (библиотека stargazer Вам поможет). 

\textbf{Задание 3}. 

Как можно обосновать предположение Гаусса-Маркова о нормальном распределении ошибок? 
\end{document}